\section{Materials and methods}
\subsection{Equations of motion: RFEM approach}

\begin{figure}[t]
    \centering
\centering\includegraphics[width=0.8 \linewidth]{Figures/rfem_mesh.eps}
    \caption{needle division to rfes and sdes.}
    \label{fig:rfem_mesh}
\end{figure}

\subsubsection{Kinematic description}
The main idea of the RFEM method is replacing flexible bodies with a system of
rigid finite elements (rfes) that are interconnected via spatial spring-damping
elements (sdes) \cite[pp. 35]{wittbrodt2007dynamics}.
The first step towards describing the kinematics of the proposed method is the
division of the flexible body into rfes and sdes. This division can be performed
in two stages \cite[pp. 37-40]{wittbrodt2007dynamics}. First, the flexible body
is divided into $n$ elements, each of which has a length $\Delta l = L / n$, where
$L$ is the needle's total length. The sdes are added at the centre point of
the initial elements, as shown in Fig. \ref{fig:rfem_mesh}. In the secondary
division, the initial elements are divided again, so that their
edge points are interconnected by the sdes. As shown in Fig.
\ref{fig:rfem_mesh}, intermediate rfes have now length equal to $\Delta L$,  while
the first and last rfe have a length of $\Delta L / 2$.
The configuration of the flexible body can now be described with the help of
the resulting $n_e$ rfes. For this,
we consider the rfe $j \in [1, n_e]$ shown in Fig. \ref{fig:rfem_element}.
Due to the rigidity assumption, the configuration of the rfe $j$ can be
completely characterised with the help of the body frame $f_{j}$, which is
rigidly attached to the element's centre of mass $C_{j}$ shown in
Fig. \ref{fig:rfem_element}. More specifically, the inertial position of an
arbitrary point $P_{j}$ of element $j$ is given as 
\begin{equation}
    \pvec{r}{OP_{j}}{F}{F} = \pvec{r}{OC_{j}}{F}{F} + \mat{R}_{f_j}^{F} 
    \pvec{r}{C_{j}P_{j}}{f_{j}}{f_{j}},
   \label{eq:rfem_position} 
\end{equation}
where $\pvec{r}{OC_{j}}{F}{F}$ is the origin of frame $f_{j}$ measured in the inertial frame $F$ and $\mat{R}_{f_j}^{F}(\vecd{\theta_{j}})$ the rotation matrix that describes the orientation of frame $f_{j}$ with respect to $F$. As before, the rotation matrix is parametrised using the set of Euler angles $\vecd{\theta}_{j}$ that describe the sequence of intrinsic rotations $z_{0}$-$y'$-$x''$. It should be noted that, due to the rigidity assumption and contrary to the DDM method, the distance $\pvec{r}{C_{j}P_{j}}{f_{j}}{f_{j}}$ is constant. Thus, a complete characterisation of the rfe $j$ configuration can be obtained using the generalised coordinates 
\begin{equation}
    \vecd{e}_{j} = \begin{bmatrix}
        (\pvec{r}{OC_{j}}{F}{F})^{T} & \vecd{\theta}_{j}^{T}
    \end{bmatrix}^{T}.
    \label{eq:rfem_genralised_coordinates}
\end{equation}
\noindent Based on expression \eqref{eq:rfem_position}, the inertial velocity of point $P_{j}$ is given as 
\begin{equation}
    \pvec{\dot{r}}{OP_{j}}{F}{F} = \mat{Z}_{j}(\vecd{x}_{j},
    \vecd{e}_{j})\ \vecd{\dot{e}}_{j}
    \quad \text{with} \quad
    \mat{Z}_{j}(\vecd{x}_{j}, \vecd{e}_{j}) = \begin{bmatrix}
        \mat{I}_{3} & -\mat{R}_{f_j}^{F} \mat{S}(\vecd{x}_{j})
        \mat{G}_{j}
    \end{bmatrix} \quad \text{and} \quad \vecd{x}_{j} = \pvec{r}{C_{j}P_{j}}{f_{j}}{f_{j}}.
    \label{eq:rfem_velocity}
\end{equation}
The matrix
$\mat{G}_{j} = \mat{G}({\vecd{\theta}_{j}})$ maps the derivatives of the Euler angles $\vecd{\dot{\theta}}_{j}$ to
the rotational velocity of frame $f_{j}$, such that 
\[
    \pvec{\omega}{f_{j}}{F}{f_{j}} = \mat{G}_{j}(\vecd{\theta}_{j})\ \vecd{\dot{\theta}}_{j}.
\]
Similarly, the inertial acceleration of point $P_{j}$ can be defined as 
\begin{equation}
    \pvec{\ddot{r}}{OP_{j}}{F}{F} = \mat{Z}_{j}(\vecd{x}_{j},
    \vecd{e}_{j})\ \vecd{\ddot{e}}_{j} + \vecd{a}_{v_{f_j}}(\vecd{x}_{j}, 
    \vecd{e}_{j}, \vecd{\dot{e}}_j),
    \label{eq:rfem_acceleration}
\end{equation}
where
\[
    \vecd{a}_{v_{f_j}}(\vecd{x}_{j}, \vecd{e}_{j}, \vecd{\dot{e}}_j) = 
    \mat{R}_{f_j}^{F} \left(\mat{S}^{2}(\pvec{\omega}{f_{j}}{F}{F})
    \ \vecd{x}_{j} - \mat{S}(\vecd{x}_{j})
    \mat{\dot{G}}_{j} \vecd{\dot{\theta}}_{j}\right).
\]   
Furthermore, based on equations \eqref{eq:rfem_position} and
\eqref{eq:rfem_velocity}, the inertial virtual displacement of point 
$P_{j}$ can be written as 
\begin{equation}
    \delta\pvec{r}{OP_{j}}{F}{F} = \frac{\partial \pvec{r}{OP_{j}}{F}{F}}{\partial
    \vecd{e}_{j}} \delta\vecd{e}_{j} =
    \mat{Z}_{j}(\vecd{x}_{j}, \vecd{e}_{j}) \delta\vecd{e}_{j}.
    \label{eq:rfem_virtual_displacement}
\end{equation}
\begin{figure}
    \centering
    \includegraphics[width=0.9\linewidth]{Figures/RFEM_Element.eps}
    \caption{Spatial RFEM element.}
    \label{fig:rfem_element}
\end{figure}

\subsubsection{Virtual work}

\paragraph{Inertial forces}

The virtual work of the inertial forces can be defined as 
\[
    \delta W_{f_j}^{(in)} = \int_{m_{f_j}} (\pvec{r}{OP_{j}}{F}{F})^{T}
    \delta \pvec{r}{OP_{j}}{F}{F}\ \mathrm{d}m_{f_j} 
\]
\noindent which, using equations \eqref{eq:rfem_acceleration} and \eqref{eq:rfem_virtual_displacement}, leads to
\begin{equation}
    \delta W_{f_j}^{(in)} = (\mat{M}_{f_j} \vecd{\ddot{e}}_{j} - \vecd{f}_{v_{f_j}})^{T}\ 
    \delta \vecd{e}_{j}
    \label{eq:rfem_virtual_work_inertial}
\end{equation}
with
\[
    \mat{M}_{f_j}(\vecd{e}_{j}) = \int_{m_{f_j}} \mat{Z}_{j}^{T}(\vecd{x}_{j}, \vecd{e}_{j})
    \mat{Z}_{j}(\vecd{x}_{j}, \vecd{e}_{j})\ \mathrm{d}m_{f_j} 
    \Rightarrow \mat{M}_{f_j} = \mat{M}_{f_j}(\vecd{e}_{j}).
\]
and 
\[
    \vecd{f}_{v_{f_j}}(\vecd{e}_{j}, \vecd{\dot{e}}_j)= - \int_{m_{f_j}} \mat{Z}_{j}^{T}(\vecd{x}_{j}, \vecd{e}_{j})\
    \vecd{a}_{v_{f_j}}(\vecd{x}_{j}, \vecd{e}_{j}, \vecd{\dot{e}}_j)\ \mathrm{d}m_{f_j}
    \Rightarrow
    \vecd{f}_{v_{f_j}} = \vecd{f}_{v_{f_j}}(\vecd{e}_{j}, \vecd{\dot{e}}_j).
\]

\paragraph{Elastic forces}

The key component of the RFEM is the definition of the elastic forces that
act on each rfe. As shown in Fig. \ref{fig:rfem_elastic_forces}, the rfe
$j$ is subject to the elastic forces $\vecd{F}_{e_{j}}, \vecd{F}_{e_{j+1}}$ and
the elastic moments $\vecd{M}_{e_{j}}, \vecd{M}_{e_{j+1}}$ due to the
deformation of the associated sdes $s_{j}$ and $s_{j+1}$.  Based on Fig.
\ref{fig:rfem_elastic_forces}, these forces and moments can be modelled as
\cite{kaminski_application_2013}
\begin{equation}
    \vecd{V}_{e_{j}}^{F} = \vecd{V}_{{el}_j}^{l} + \vecd{V}_{d_j}^{l}
    \quad \text{and} \quad \vecd{V}_{e_{j+1}}^{F} = \vecd{V}_{{el}_j}^{r} +
    \vecd{V}_{d_j}^{r} \quad \text{with} \quad V=\{F, M\}.
    \label{eq:rfem_elastic_forces_moments}
\end{equation}
The indices $el$ and ${d}$ refer to spring and damping forces/moments
respectively. These can be modelled as
\begin{equation}
    \begin{bmatrix}
        \vecd{F}_{{el}_j}^{\omega} \\
        \vecd{M}_{{el}_j}^{\omega} \\
        \vecd{F}_{d_j}^{\omega} \\
        \vecd{M}_{d_j}^{\omega}
    \end{bmatrix} = 
    \begin{bmatrix}
        \mat{K}_{s} & \mat{O}_{3} & \mat{O}_{3} & \mat{O}_{3}\\
        \mat{O}_{3} & \mat{K}_{t} & \mat{O}_{3} & \mat{O}_{3} \\
        \mat{O}_{3} & \mat{O}_{3} & \mat{C}_{s} & \mat{O}_{3} \\
        \mat{O}_{3} & \mat{O}_{3} & \mat{O}_{3} & \mat{C}_{t}
    \end{bmatrix}
    \begin{bmatrix}
        \vecd{r}_{\omega_j} \\
        \vecd{\theta}_{\omega_j} \\
        \vecd{\dot{r}}_{\omega_j} \\
        \vecd{\dot{\theta}}_{\omega_j}
    \end{bmatrix}
    \label{eq:elastic_forces}
\end{equation}
\noindent with $\omega=\{l, r\}$ and $\mat{O}_{3}$ the $3\times 3$ null matrix.
. Expressions for the constant stiffness and damping matrices $\mat{K}_{s}, \mat{C}_{s}, \mat{K}_{t}$ and $\mat{C}_{t}$ can be found in \cite[pp. 69-82]{wittbrodt2007dynamics}. In the above expression, vectors $\vecd{r}_{r_j}$ and $\vecd{r}_{l_j}$ denote the translational deformation of the sdes $s_j$ and $s_{j+1}$ respectively, and can be written as
\[
    \vecd{r}_{l_j} = \pvec{r}{OA_{j}}{F}{F} - \pvec{r}{OB_{j-1}}{F}{F}
    \quad \text{and} \quad
    \vecd{r}_{r_j} = \pvec{r}{OA_{j+1}}{F}{F} - \pvec{r}{OB_{j}}{F}{F}.
\]
Similarly, the elastic moments in equation \eqref{eq:rfem_elastic_forces_moments} are modelled based on the rotational deformations of the associated sdes $s_{j}$ and $s_{j+1}$, denoted as $\vecd{\theta}_{l_j}$ and $\vecd{\theta}_{r_j}$. These can be extracted from the rotation matrices $\mat{R}_{f_{j}}^{f_{j-1}}(\vecd{\theta}_{l_j})$ and $\mat{R}_{f_{j+1}}^{f_{j}}(\vecd{\theta}_{r_j})$ accordingly. From the expression above, the element's generalised elastic forces are nonlinear functions of the system's generalised coordinates. As a result, if these expressions are treated as forces external to the element $j$, then the virtual work caused by the generalised elastic and damping forces can be written as 
\begin{equation}
    \delta W_{f_j}^{(el)} = \vecd{Q}_{{el}_{j}}^{T} \delta \vecd{e}_{j} 
    \quad \text{and} \quad 
    \delta W_{f_j}^{(d)} = \vecd{Q}_{{d}_{j}}^{T} \delta \vecd{e}_{j},
    \label{eq:rfem_virtual_work_elastic}
\end{equation}
where
\begin{equation}
    \vecd{Q}_{{\omega}_{j}}^{T} = \mat{Z}_{B_{j}}^{T} \vecd{F}_{{\omega}_j}^{r} - 
    {\mat{Z}}_{A_{j}}^{T} \vecd{F}_{{\omega}_{j}}^{l} +
    \left(\mat{R}_{f_j}^{F} \mat{G}_{j} \mat{L}_{\theta}\right)^{T} \left(
    \vecd{M}_{{\omega}_j}^{r} - \vecd{M}_{{\omega}_j}^{l} \right)
    \quad \text{with} \quad  \omega= \{el, d\}
    \label{eq:generalized_elastic_forces}
\end{equation}
and
\begin{equation}
    \mat{Z}_{D_{j}} = \begin{bmatrix}
        \mat{I}_{3} & -\mat{R}_{f_j}^{F}
        \mat{S}(\pvec{r}{C_{j}D_{j}}{f_{j}}{f_{j}}) \mat{G}_{j}
    \end{bmatrix} \quad \text{with} \quad D=\{A, B\}.
    \label{eq:generalized_elastic_forces_z_matrix}
\end{equation}
The locator matrix $\mat{L}_{\theta}$ in the above expression maps the element's Euler angles to the element's generalised coordinates such that $\vecd{\theta}_{j} = \mat{L}_{\theta}\vecd{e}_{j}$.

\subparagraph{Boundary Elements}
As shown in Fig. \ref{fig:rfem_elastic_forces_boundaries}, the elastic forces
for the boundary elements (${j=1}$ and $j={n_e}$)
cannot be defined the same way as in \eqref{eq:generalized_elastic_forces}.
For these elements the elastic forces are defined as 
\begin{figure}
    \centering\includegraphics[width=0.8\linewidth]{Figures/RFEM_Elastic.eps}
    \caption{Elastic forces in RFEM.}
    \label{fig:rfem_elastic_forces}
\end{figure}
\begin{figure}
    \centering\includegraphics[width=0.75\linewidth]{Figures/RFEM_Elastic_Boundaries.pdf}
    \caption{Elastic forces in the boundary elements of RFEM.}
    \label{fig:rfem_elastic_forces_boundaries}
\end{figure}
\[
    \vecd{Q}_{{\omega}_{j}}^{T} = \mat{Z}_{B_{j}}^{T} \vecd{F}_{{\omega}_j}^{r} +
    \left(\mat{R}_{f_j}^{F} \mat{G}_{j} \mat{L}_{\theta}\right)^{T}
    \vecd{M}_{{\omega}_j}^{r}
    \quad \text{with} \quad  \omega= \{el, d\} \quad \text{and} \quad j=1,
\]
and 
\[
    \vecd{Q}_{{\omega}_{j}}^{T} = - {\mat{Z}}_{A_{j}}^{T}
    \vecd{F}_{{\omega}_{j}}^{l} - \left(\mat{R}_{f_j}^{F}
    \mat{G}_{j} \mat{L}_{\theta}\right)^{T} \vecd{M}_{{\omega}_j}^{l}
    \quad \text{with} \quad  \omega= \{el, d\} \quad \text{and} \quad j={n_e}.
\]
The definition of the matrices ${\mat{Z}}_{A_{j}}$ and ${\mat{Z}}_{B_{j}}$ 
is identical to \eqref{eq:generalized_elastic_forces_z_matrix}.

\paragraph{External Forces}
Next, the virtual work of the external forces is defined. For this, we consider
the point $P^{i}_{j}$ of element $j$, located at the position defined by the
vector $\vecd{x}_{j}^{i}$ with respect to frame $f_{j}$, which is subject to
the point force and point moment pair illustrated in Fig.
\ref{fig:rfem_external_forces}.

\[
    \delta W_{f_j}^{(ex)} = (\prescript{j}{}{\vecd{F}^{F}_{i}})^{T} \delta 
    \pvec{r}{OP_{j}^{i}}{F}{F} + (\prescript{j}{}{\vecd{M}^{F}_{i}})^{T}
    \mat{R}_{f_j}^{F} \mat{G}_{j} \delta \vecd{\theta}_{j}^{i}.
\]
Using equation \eqref{eq:rfem_virtual_displacement} and the matrix $\mat{L}_{\theta}$, the above expression can be written as
\begin{equation}
    \delta W_{f_j}^{(ex)} = \left[(\prescript{j}{}{\vecd{F}^{F}_{i}})^{T}
    \mat{Z}_{j}^{i} + (\prescript{j}{}{\vecd{M}^{F}_{i}})^{T}
    \mat{R}_{f_j}^{F} \mat{G}_{j} \mat{L}_{\theta} \right] \delta \vecd{e}_{j},
    \label{eq:rfem_virtual_work_external_one_force}
\end{equation}
where $\mat{Z}_{j}^{i} = \mat{Z}(\vecd{x}_{j}^{i} , \vecd{e}_{j})$. If we consider that the element $j$ is subject to $n_{f}$ point forces and $n_m$ point moments,
it can be shown that the total virtual work of the external generalised forces is given as
\begin{equation}
    \delta W_{f_j}^{(ex)} = \vecd{Q}_{f_j}^{T} \delta \vecd{e}_{j}
    \quad \text{with} \quad 
    \vecd{Q}_{f_j} = \sum_{i = 1}^{n_f} (\mat{Z}_{j}^{i})^{T}
    (\prescript{j}{}{\vecd{F}^{F}_{i}}) +
    \mat{L}_{\theta}^{T} \mat{G}_{j} \mat{R}_{F}^{f_{j}}\sum_{k = 1}^{n_m} 
    (\prescript{j}{}{\vecd{M}^{F}_{k}}).
    \label{eq:rfem_virtual_work_external}
\end{equation}

\begin{figure}
    \centering\includegraphics[width=0.4\linewidth]{Figures/RFEM_Forces.eps}
    \caption{Free body diagram of element $j$ for the RFEM method.}
    \label{fig:rfem_external_forces}
\end{figure}


\subsection{Dynamic Equilibrium}
To derive the flexible needle's equations of motion we apply the dynamic equilibrium expression based on D'Alembert's form of the principle of virtual work, which can be written as
\[
    \delta W_{f}^{(in)} = \delta W_{f}^{(el)} + 
    \delta W_{f}^{(d)} +  \delta W_{f}^{(ex)} + \delta W_{f}^{(c)}.
\]
The virtual works of the above expression can be calculated considering the contributions from the individual elements as
\[
    \sum_{j=1}^{n_{e}} \delta W_{f_{j}}^{(in)} = \sum_{j=1}^{n_{e}} \left(\delta W_{f_{j}}^{(el)} + 
    \delta W_{f_{j}}^{(d)} +  \delta W_{f_{j}}^{(ex)} + \delta W_{f_{j}}^{(c)}
    \right).
\]
Substituting the derived virtual work expressions, the above equation becomes
\[
    \sum_{j=1}^{n_{e}} \left(\mat{M}_{f_{j}}\vecd{\ddot{e}}_{j} - \vecd{f}_{v_{f_j}} 
     - \vecd{Q}_{el_j} - \vecd{Q}_{d_j} - \vecd{Q}_{f_j} - \vecd{Q}_{c_j} \right)^{T} \delta \vecd{e}_{j} = 0.
\]
If we introduce the mapping from the local generalised coordinates 
$\vecd{e}_{j}$ of element $j$ to the flexible beam global generalised coordinates
$\vecd{\bar{q}} $$\in \mathbb{R}^{n_q}$ , such that $\vecd{e}_{j} = \mat{L}_{j} \vecd{\bar{q}}$, then the above expression can be written as
\begin{equation}
    (\mat{M}_{f}\ \vecd{\ddot{\bar{q}}} - \vecd{f}_{v_f} - \vecd{Q}_{el} -
    \vecd{Q}_{d} - \vecd{Q}_{f} - \vecd{Q}_{c})^{T} \delta\vecd{\bar{q}} = 0,
\end{equation}
with
\small
\begin{equation}
    \mat{M}_{f} = \sum\limits_{j=1}^{n_{e}}\mat{L}_{j}^{T} \mat{M}_{f_j} \mat{L}_{j}, 
    \quad 
    \vecd{f}_{v_f} = \sum\limits_{j=1}^{n_{e}}\mat{L}_{j}^{T} \vecd{f}_{v_{f_j}}
    \quad \text{and} \quad 
    \vecd{Q}_{\omega} = \sum\limits_{j=1}^{n_{e}}\mat{L}_{j}^{T} \vecd{Q}_{\omega_{j}}
    \quad \text{with} \quad \omega = \{el, d, f, c \}.
    \label{eq:matrices_sums}
\end{equation}
\normalsize
Due to the explicit integration of the constraint forces $\vecd{Q}_{c}$ in the
above expression, the generalised coordinates of the flexible body can be
treated as independent and, thus, its equations of motion can be expressed as
\begin{equation}
    \mat{M}_{f}\ \vecd{\ddot{\bar{q}}} = \vecd{f}_{v_f} + \vecd{Q}_{el} +
    \vecd{Q}_{d} + \vecd{Q}_{f} + \vecd{Q}_{c}.
    \label{eq:equations_of_motion}
\end{equation}
or
\begin{equation}
    \mat{M}_{f}(\vecd{\bar{q}})\ \vecd{\ddot{\bar{q}}} =
    \vecd{f}_{v_f}(\bar{\vecd{q}}, \dot{\bar{\vecd{q}}}) +
    \vecd{Q}_{el}(\bar{\vecd{q}}) + \vecd{Q}_{d}(\bar{\vecd{q}}, \dot{\bar{\vecd{q}}})
    + \vecd{Q}_{f}(t, \bar{\vecd{q}}, \dot{\bar{\vecd{q}}}) +
    \vecd{Q}_{c}.
    \label{eq:equations_of_motion_complete}
\end{equation}


\subsection{Constraint forces}
The state vector $\vecd{\bar{q}}$
contains the boundary conditions, due to the rigid connection between the
flexible needle and the rigid base. The constrained vector $\vecd{q}_{c}$
incorporates the prescribed motion of the associated coordinates
(boundary conditions), while the active vector $\vecd{q}_{a}$ describes the
free (independent) coordinates that define the beam's motion. For the
rigid finite element method the constrained vector
$\vecd{q}_{c}$ can take the following form

\begin{equation}
    \vecd{\bar{q}} = \begin{bmatrix}
        \vecd{q}_{c} \\ \vecd{q}_{a}
    \end{bmatrix}, \quad \text{with} \quad
    \vecd{q}_{c}=  \begin{bmatrix}
        \pvec{r}{OC_{1}}{F}{F} \\ \vecd{\theta}_{1}.
    \end{bmatrix}
    \label{eq:state_partitioning}
\end{equation}

In a similar fashion, we can split the constrained forces $\vecd{Q}_{c}$
to their constrained and active componenets ${\vecd{Q}_{c_c}}$ and 
$\vecd{Q}_{c_a}$. The key idea for obtaining the system's solution is
that the components of the constraint
forces corresponding to the active system's coordinates
are zero, $\vecd{Q}_{c_a} = \vecd{0}_{n_a}$.
On the other hand, the components $\vecd{Q}_{c_c}$, which result
from the reaction forces due to
the boundary conditions at point $A$, are finite and incorporate information
about the connection forces $\vecd{F}_{A}$ and moments
$\vecd{M}_{A}$. They also constitute the only variables that the 
doctor or the robotic system can actively control to steer the 
system to the desired trajectory (control input).

The generalized contraint forces $Q_{c_c}$ can be correlated with the
forces acting on the first element the RFEM system as
\[
    \vecd{Q}_{c_c} = - \begin{bmatrix}
    \mat{R}_{f_1}^{F} \vecd{F}_{C_{1}}^{f_{1}} \\
    \mat{G}_{1}^{T} \vecd{M}_{C_{1}}^{f_{1}}
    \end{bmatrix} = \mat{D}(\vecd{\bar{q}}) \vecd{u}, \quad \text{with} \quad 
    \mat{D}(\vecd{\bar{q}})
    \begin{bmatrix}
        \mat{R}_{f_1}^{F} && \mat{O}_{3} \\
        \mat{O}_{3} && \mat{G}_{1}^{T}
    \end{bmatrix}
    \quad \text{and} \quad \vecd{u} = \begin{bmatrix}
    \vecd{F}_{C_{1}}^{f_{1}} \\
    \vecd{M}_{C_{1}}^{f_{1}}
    \end{bmatrix}.
\]

The vectors $\vecd{F}_{C_{1}}^{f_{1}}$ and $\vecd{M}_{C_{1}}^{f_1}$ represent the 
net force and moment acting on the rfe $1$ with respect to its
body frame respectively. The total vector $Q_{c}$ is then written as 
\[
    \vecd{Q}_{c} = \begin{bmatrix}
        \vecd{Q}_{c_c} \\ 
        \vecd{Q}_{c_a} 
    \end{bmatrix}  =
    \begin{bmatrix}
        \mat{D}(\vecd{\bar{q}}) \vecd{u} \\ 
        \mat{O}_{n_{a} \times n_{u}}
    \end{bmatrix}  =
    \begin{bmatrix}
        \mat{D}(\vecd{\bar{q}}) \\ 
        \mat{O}_{n_{a} \times n_{u}}
    \end{bmatrix} \vecd{u} = \mat{\hat{D}}(\vecd{\bar{q}})\vecd{u}, \quad \text{with}
    \quad \mat{\hat{D}}(\vecd{\bar{q}}) = 
    \begin{bmatrix}
        \mat{D}(\vecd{\bar{q}}) \\ 
        \mat{O}_{n_{a} \times n_{u}}
    \end{bmatrix}.
\]

Based on the above, the equations of motion can be written as 

\begin{equation}
    \mat{M}_{f}(\vecd{\bar{q}})\ \vecd{\ddot{\bar{q}}} =
    \vecd{f}_{v_f}(\bar{\vecd{q}}, \dot{\bar{\vecd{q}}}) +
    \vecd{Q}_{el}(\bar{\vecd{q}}) + \vecd{Q}_{d}(\bar{\vecd{q}}, \dot{\bar{\vecd{q}}})
    + \vecd{Q}_{f}(t, \bar{\vecd{q}}, \dot{\bar{\vecd{q}}}) +
    \mat{\hat{D}}(\vecd{\bar{q}})\vecd{u}.
    \label{eq:equations_of_motion_with_control}
\end{equation}

\subsection{State space reperesentation}

Define $\vecd{x}_{1} = \vecd{\bar{q}}$ and $\vecd{x}_{2} = \vecd{\dot{x}}_{1} = 
\vecd{\dot{\bar{q}}}$. Then equation
\eqref{eq:equations_of_motion_with_control} can be written as 

\[
    \mat{M}_{f}(\vecd{x}_{1})\ \vecd{\dot{x}}_{2} =
    \vecd{f}_{v_f}(\vecd{x}_{1}, \vecd{x}_{2}) +
    \vecd{Q}_{el}(\vecd{x}_{1}) + \vecd{Q}_{d}(\vecd{x}_1, \vecd{x}_{2})
    + \vecd{Q}_{f}(t, \vecd{x}_1, \vecd{x}_{2}) +
    \mat{\hat{D}}(\vecd{x}_{1})\vecd{u}.
\]
or

\begin{multline}
    \ \vecd{\dot{x}}_{2} =
    \mat{M}^{-1}_{f}(\vecd{x}_{1})
    \left(\vecd{f}_{v_f}(\vecd{x}_{1}, \vecd{x}_{2}) +
    \vecd{Q}_{el}(\vecd{x}_{1}) + \vecd{Q}_{d}(\vecd{x}_1, \vecd{x}_{2})
    + \vecd{Q}_{f}(t, \vecd{x}_1, \vecd{x}_{2})\right) \\
    + \mat{M}^{-1}_{f}(\vecd{x}_{1}) 
    \mat{\hat{D}}(\vecd{x}_{1})\vecd{u}.
\end{multline}

The complete state space representation of the system is then defined as 
\begin{equation}
   \vecd{\dot{x}} = \vecd{f}(t, \vecd{x}) + \mat{B}(\vecd{x})\vecd{u}, 
   \quad \text{with} \quad \vecd{x} = \begin{bmatrix}
        \vecd{x}_{1} \\ 
        \vecd{x}_{2}
   \end{bmatrix} \in \mathbb{R}^{n}.
\end{equation}

The system's model function $\vecd{f}(t, \vecd{x})$ is given 
as
\[
    \vecd{f}(t, \vecd{x}) = \begin{bmatrix}
        \vecd{x}_{2} \\
    \mat{M}^{-1}_{f}(\vecd{x}_{1})
    \left(\vecd{f}_{v_f}(\vecd{x}_{1}, \vecd{x}_{2}) +
    \vecd{Q}_{el}(\vecd{x}_{1}) + \vecd{Q}_{d}(\vecd{x}_1, \vecd{x}_{2})
    + \vecd{Q}_{f}(t, \vecd{x}_1, \vecd{x}_{2})\right)
    \end{bmatrix},
\]

while the control matrix $\mat{B}$ can be written as 
\[
    \mat{B}(\vecd{x}) = \begin{bmatrix}
        \mat{O}_{n_{q} \times n_{u}} \\
    \mat{M}^{-1}_{f}(\vecd{x}_{1}) 
    \mat{\hat{D}}(\vecd{x}_{1})
    \end{bmatrix}.
\]




    % \left(\mat{M}_{f}(\vecd{x}_{1})\right)^{-1}



% // Get reaction force at C0
% arma::dvec fc0_F = -qcc_force.rows(0, 2);

% // Get reaction moment at C0
% arma::dvec mc0_f = -arma::solve(g0_mat.t(), qcc_force.rows(3, 5));


% with the forces 
% acting on the system we introduce the following equation


% \eqref{eq:equations_of_motion_matrix} can be written as

% \[
%     \vecd{Q}_{c} = \begin{bmatrix}
%         \vecd{Q}_{c_c} \\
%         \vecd{Q}_{c_a}
%     \end{bmatrix}   
% \]

% \subsection{Coordinate Partitioning}
% Before the proposed algorithm is formulated, some important remarks
% for equation \eqref{eq:equations_of_motion} must be made. More
% specifically, it should be noted that the state vector $\vecd{\bar{q}}$
% contains the boundary conditions, due to the rigid connection between the
% flexible needle and the rigid base. The constrained vector $\vecd{q}_{c}$
% incorporates the prescribed motion of the associated coordinates
% (boundary conditions), while the active vector $\vecd{q}_{a}$ describes the
% free (independent) coordinates that define the beam's motion. For the
% rigid finite element method the constrained vector
% $\vecd{q}_{c}$ can take the following form

% \begin{equation}
%     \vecd{\bar{q}} = \begin{bmatrix}
%         \vecd{q}_{c} \\ \vecd{q}_{a}
%     \end{bmatrix}, \quad \text{with} \quad
%     \vecd{q}_{c}=  \begin{bmatrix}
%         \pvec{r}{OC_{1}}{F}{F} \\ \vecd{\theta}_{1}
%     \end{bmatrix} \quad \text{and} \quad
%     \vecd{q}_{a}=  \begin{bmatrix}
%         \pvec{r}{OC_{i}}{F}{F} \\ \vecd{\theta}_{i}
%     \end{bmatrix}, i = \{2, \dots n_{e} \}
%     \label{eq:state_partitioning}
% \end{equation}

% Base on the above, the equations of motion
% \eqref{eq:equations_of_motion},
% can be rewritten as
% \begin{equation}
%     \begin{bmatrix}
%        \mat{M}_{f_{cc}} & \mat{M}_{f_{ca}}  \\
%        \mat{M}_{f_{ac}} & \mat{M}_{f_{aa}}  \\
%     \end{bmatrix} \begin{bmatrix}
%        \vecd{\ddot{q}}_{c} \\
%        \vecd{\ddot{q}}_{a}
%     \end{bmatrix} = 
%     \begin{bmatrix}
%         \vecd{g}_{c}(t, \vecd{\bar{q}}, \vecd{\dot{\bar{q}}}) \\
%         \vecd{g}_{a}(t, \vecd{\bar{q}}, \vecd{\dot{\bar{q}}})
%     \end{bmatrix},
%     \label{eq:equations_of_motion_matrix}
% \end{equation}

% where
% \[
%     \begin{bmatrix}
%         \vecd{g}_{c} \\
%         \vecd{g}_{a}
%     \end{bmatrix} = \begin{bmatrix}
%         \vecd{f}_{v_{f_c}} \\
%         \vecd{f}_{v_{f_a}}
%     \end{bmatrix} + \begin{bmatrix}
%         \vecd{Q}_{el_{c}} \\
%         \vecd{Q}_{el_{a}}
%     \end{bmatrix}+ \begin{bmatrix}
%         \vecd{Q}_{d_{c}} \\
%         \vecd{Q}_{d_{a}}
%     \end{bmatrix} + \begin{bmatrix}
%         \vecd{Q}_{f_{c}} \\
%         \vecd{Q}_{f_{a}}
%     \end{bmatrix}+ \begin{bmatrix}
%         \vecd{Q}_{c_{c}} \\
%         \vecd{Q}_{c_{a}}
%     \end{bmatrix}.
% \]

% Note that for a given initial state of the system the components of the vector
% $\vecd{g} = [\vecd{g}_{c}^{T}, \vecd{g}_{a}^{T}]^{T}$ are all known, except for
% the generalised constrained forces $\vecd{Q}_{c}$.
% These generalized contrained forces constitute the only input to the system.
% The key idea for
% obtaining the system's solution is that the components of the constraint
% forces corresponding to the active system's coordinates
% are zero, $\vecd{Q}_{c_a} = \vecd{0}_{n_a}$.
% On the other hand, the
% components $\vecd{Q}_{c_c}$, which result from the reaction forces due to
% the boundary conditions at point $A$, are finite and incorporate information
% about the connection forces $\vecd{F}_{A}$ and moments
% $\vecd{M}_{A}$. Based on this observation, equation
% \eqref{eq:equations_of_motion_matrix} can be written as

% \begin{equation}
%     \mat{M}_{f}(\vecd{\bar{q}})\ \vecd{\ddot{\bar{q}}} =
%     \vecd{f}_{v_f}(\bar{\vecd{q}}, \dot{\bar{\vecd{q}}}) +
%     \vecd{Q}_{el}(\bar{\vecd{q}}) + \vecd{Q}_{d}(\bar{\vecd{q}}, \dot{\bar{\vecd{q}}})
%     + \vecd{Q}_{f}(t, \bar{\vecd{q}}, \dot{\bar{\vecd{q}}}) +
%     \mat{B}\vecd{u}
% \end{equation}




% \subsection{Mass matrix properties}
% The mass matrix for element ${j}$ is given as
% \[
%     \mat{M}_{f_j}(\vecd{e}_{j}) = \int_{m_{f_j}} \mat{Z}_{j}^{T}(\vecd{x}_{j}, \vecd{e}_{j})
%     \mat{Z}_{j}(\vecd{x}_{j}, \vecd{e}_{j})\ \mathrm{d}m_{f_j} 
% \]

% or

% \[
%     \mat{M}_{f_j}(\vecd{e}_{j}) = \int_{m_{f_j}} \begin{bmatrix}
%         \mat{I}_{3} &&
%         -\mat{R}_{f_j}^{F} \mat{S}(\pvec{r}{C_{j}P_{J}}{f_j}{f_j})\mat{G}_{j} \\ 
%         \left(-\mat{R}_{f_j}^{F} \mat{S}(\pvec{r}{C_{j}P_{J}}{f_j}{f_j})\mat{G}_{j})\right)^{T} &&
%         \mat{G}_{j}^{T} \mat{S}^{T}(\pvec{r}{C_{j}P_{J}}{f_j}{f_j})
%         \mat{S}(\pvec{r}{C_{j}P_{J}}{f_j}{f_j}) \mat{G}_{j} 
%     \end{bmatrix}\mathrm{d}m_{f_j}
% \]

% or 

% \[
%     \mat{M}_{f_j}(\vecd{e}_{j}) = \int_{m_{f_j}} \begin{bmatrix}
%         m_{f_j} \mat{I}_{3} && \mat{O}_{3} \\
%         \mat{O}_{3} && \mat{G}_{j}^{T} \mat{I}_{C_j}^{f_j} \mat{G}_{j} \\
%     \end{bmatrix}\mathrm{d}m_{f_j}.
% \]